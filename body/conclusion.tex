\hspace{24pt}

% 我們將總結整個論文所做的假設、方法、貢獻。講出本論文能進步、加強的地方。

% 這個章節首先將在a總結論文所提出的假設,方法,及貢獻。接著在b提出一系列可以深入研究或改進的地方,作為我們的未來規劃。

\section{Conclusion} \label{sec:conclusion}

% 假設、方法、貢獻

% 本篇論文嘗試使用聲音資訊作為文字的額外特徵,並將該特徵應用於中文與日文之間的神經機器翻譯系統當中,旨在透過常規以外的方法如 back translation 或 model ensembling ,以 feature engineering 的方式來加強翻譯系統。 我們提出使用三種不同的分詞方法,以 dragonmapper 及 pykakasi 分別產生中文及日文的聲音資訊,並搭配 word2vec 訓練出兩個單純帶有語意和音意的詞嵌入。我們合併兩者產生帶有語意及音意的合併詞嵌入,並將其投入至兩種知名的 NMT 模型進行訓練與fine-tune。結果表明所有使用了合併詞嵌入進行訓練與fine-tune的模型,其獲得的 BLEU 分數都比其他方法還要更高。我們透過對最好模型的結果進行 case study 發現使用合併詞嵌入翻譯的結果保留了四種property。我們透過對合併詞嵌入與單純包含詞義的嵌入進行分析,結果也表明在四種任務下皆保留原有效果,甚至能超越詞義嵌入,獲得非常正面的反饋。總結,該論文提出的方法能有效提升nmt效能,而且僅使用小型語料庫。對於翻譯結果與詞嵌入我們也展示一系列的分析,得知該詞嵌入確實對於中日文翻譯有明顯的幫助。

\section{Future Work} \label{sec:future_work}

% 本論文有許多能夠深入研究的方向,我們在此列出幾個方法,這些方法有的或許能進一步分析現有表現,有的或許有機會能進一步提高表現。

% 改善產生聲音資訊的方法,例如使用其他的插件來產生並比較優劣,舉例

% 採用在 related work 中提到訓練 embedding 的模式,將聲音資訊與中文字、及各種 subcharacter feature 例如筆畫、部首來一起訓練一個 embedding。

% 採用 ELMO、BERT 當作基礎的 embedding 架構。由於他們兩者都是屬於需要綁定模型的 embedding,所以可以考慮訓練一個 semantic model 和一個 phonetic model 並將兩者結合起來。

% 最新研究展示 cnn 用於 NLP 領域強於 transformer。所以我們也可以試用 CONVS2S 當作模型架構,。