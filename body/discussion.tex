\hspace{24pt}

% 在 case study 我們會挑選一些翻譯結果,試著解釋為什麼翻譯較好的原因。
% 在 embedding analysis 我們探討 joint 和 semantic 比對的結果。

\section{Case Study} \label{sec:case_study}

% 列個 5 種吧。講解哪些詞翻較好。哪些地方比較成功。用聲音資訊來解釋佳。

% \begin{center}
%     \begin{CJK*}{UTF8}{gbsn}
%     \begin{tabular}{p{7.3cm}p{27.5cm}} \toprule
%       Source & 微小粒子测量装置的比较试验\\\midrule
%     \end{tabular}
%     \end{CJK*}
%     \begin{CJK}{UTF8}{min}
%     \begin{tabular}{p{7.3cm}p{27.5cm}}
%       Target & 微小粒子測定装置の比較試験\\
%       Semantic & 微小粒子計測装置の比較実験 \\
%       Semantic + Phonetic & 微小粒子測定装置の比較試験\\
%     \end{tabular}
%     \end{CJK}
%     \begin{CJK*}{UTF8}{gbsn}
%     \begin{tabular}{p{7.3cm}p{27.5cm}} \toprule
%       Source & 从背景知识B和观测结果O中获得行动规则的集合Y的集合H。\\\midrule
%     \end{tabular}
%     \end{CJK*}
%     \begin{CJK}{UTF8}{min}
%     \begin{tabular}{p{7.3cm}p{27.5cm}}
%       Target & 背景知識Bと観測結果Oより行動規則の集合Yの集合を獲得する.\\
%       Semantic & 背景背景知識Bと観測結果Oから行動ルールの集合Yの集合H獲得する. \\
%       Semantic + Phonetic & 背景知識Bと観測結果Oから行動規則の集合Yの集合H獲得する.\\\bottomrule
%     \end{tabular}
%     \end{CJK}
%     \captionof{table}{\color{Green}Case Study}
%     \label{table:case_study}
%     \end{center}

\section{Embedding Analysis} \label{sec:analysis}

% 以四種方式來探討 joint embedding 的優點,與 semantic 的差異。

\subsection{Analogical Reasoning} \label{sec:analysis_analogy}

% 放上結果,講差別,不只保留字義,且超過,優點

% \begin{center}\vspace{0.3cm}
%     \begin{CJK*}{UTF8}{bsmi}
%         \begin{tabular}{P{3.5cm} P{8cm} P{8cm} P{8cm}}
%             \toprule
%             \multirow{2.5}{*}{Language} & \multirow{2.5}{*}{Input ($a:A=b:$)} & \multicolumn{2}{c}{Output ($B$)} 
%             \\\cmidrule{3-4}
%             & & Semantic Only & Semantic + Phonetic \\\midrule
%             Chinese & 東京:日本$=$北京: & 中國 \ ($p=0.49$) & 中國 \ ($p=0.53$) \\
%             {} & 長期:三年$=$短期: & 一年 \ ($p=0.38$) & 兩周 \ ($p=0.39$) \\
%             {} & 進口:買入$=$出口: & 賣出 \ ($p=0.36$) & 賣出 \ ($p=0.44$) \\\midrule
%         \end{tabular}
%     \end{CJK*}
    
%     \begin{CJK}{UTF8}{min}
%         \begin{tabular}{P{3.5cm} P{8cm} P{8cm} P{8cm}}
%             Japanese & 男性:女性$=$父親: & 母親 \ ($p=0.49$) & 母親 \ ($p=0.51$) \\
%             & 長期:年$=$短期: & 月 \ ($p=0.55$) & 月 \ ($p=0.57$) \\
%             & 左右:前後$=$水平: & 垂直 ($p=0.43$) & 垂直 ($p=0.40$) \\
%             \bottomrule
%         \end{tabular}
%     \end{CJK}
    
%     \captionof{table}{\color{Green}Analogy Test}
%     \label{table:analogy_test}
% \end{center}


\subsection{Outlier Detection} \label{sec:analysis_outlier}

% 放上結果,講差別,優點

% \begin{center}\vspace{0.3cm}
%     \begin{CJK*}{UTF8}{bsmi}
%         \begin{tabular}{P{3.5cm} P{12cm} P{6cm} P{8cm}}
%             \toprule
%             \multirow{2.5}{*}{Language} & \multirow{2.5}{*}{Input} & \multicolumn{2}{c}{Output (Outlier)} 
%             \\\cmidrule{3-4}
%             & & Semantic Only & Semantic + Phonetic \\\midrule
            
%             Chinese & 維持, 保持, 堅持, 建設  & 建設 & 建設 \\
%             {} & 可行, 不行, 可以, 行 & 可以 & 不行 \\\midrule
            
%         \end{tabular}
%     \end{CJK*}
    
%     \begin{CJK}{UTF8}{min}
%         \begin{tabular}{P{3.5cm} P{12cm} P{6cm} P{8cm}}
%         Japanese & 生み, 創造, 作る, 破壊 & 破壊 & 破壊\\
%         & 普通, 一般, 通常, 異常 & 異常 & 異常\\
%         \bottomrule
%         \end{tabular}
%     \end{CJK}
    
%     \captionof{table}{\color{Green}Outlier Detection}
%     \label{table:outlier_detection}
% \end{center}


\subsection{Word Similarity} \label{sec:analysis_similarity}

% 放上結果,講差別,優點

\subsection{Homonym and Heteronym} \label{sec:analysis_homonym_heteronym}

% 放上結果,講差別,優點



% \begin{center}\vspace{0.3cm}
%     \begin{CJK*}{UTF8}{bsmi}
%         \begin{tabular}{P{3.5cm} P{12cm} P{6cm} P{8cm}}
%             \toprule
%             \multirow{2.5}{*}{Language} & \multirow{2.5}{*}{Input} & \multicolumn{2}{c}{Distance} 
%             \\\cmidrule{3-4}
%             & & Semantic Only & Semantic + Phonetic \\\midrule
%             Chinese & 長(ㄔㄤˊ)度, 長(ㄓㄤˇ)大  & 0.826 & \textbf{0.853} \\
%             {} & 樂(ㄌㄜˋ)趣, 音樂(ㄩㄝˋ) & 0.636 & \textbf{0.682} \\
%             {} & 中(ㄓㄨㄥ)午, 中(ㄓㄨㄥˋ)毒 & 0.841 & \textbf{0.866}\\\midrule
%         \end{tabular}
%     \end{CJK*}
%     \begin{CJK}{UTF8}{min}
%         \begin{tabular}{P{3.5cm} P{12cm} P{6cm} P{8cm}}
%                 Japanese & 生(なま), 一生(しょう)  & 0.879 & \textbf{0.889} \\
%                 {} & 生(なま), 生(き)地 & 0.769 & \textbf{0.867} \\
%                 {} & 生(なま), 生(う)む & 0.829 & \textbf{0.839} \\\bottomrule
%         \end{tabular}
%     \end{CJK}
%     \captionof{table}{\color{Green}Similarity Distances Between Heteronyms}
%     \label{table:noise_heteronym}
% \end{center}