\hspace{24pt}

% 方法加個圖吧

% 講解整個實驗的流程。用什麼資料集。用什麼環境展開實驗。模型中用什麼參數。用什麼評分翻譯系統。所有翻譯系統的評分結果。

\section{Dataset} \label{sec:dataset}

% 講解 Dataset 的來源。內容。架構。

\section{Environment} \label{sec:environment}

% 講解執行 NMT 實驗會用到的架構和 logger。

\subsection{PyTorch Lightning} \label{sec:lightning}

% lightning 的理念,核心,好處。

\subsection{wandb} \label{sec:wandb}

% wandb 的理念,核心,好處。

\section{Parameter} \label{sec:parameter}

% 分別在兩種模型的一般參數使用哪些數值。以及 hyperparameter。還有 embedding 解凍。

\section{Metric} \label{sec:metric}

% 我們用 BLEU 做為評分標準。BLEU 是什麼,怎麼算。常見的分數解釋。

\section{Result} \label{sec:result}

% 我們以 WAT 2020 做為標準。什麼是 WAT 2020。他們同樣使用 ASPEC。分別用什麼分詞。分數為多少。

% RNN 的結果,SENTENCEPIECE 結果,LOSS,BLEU。JIEBA JANOME 結果,LOSS,BLEU。

% TRANSFORMER 的結果,SENTENCEPIECE 結果,LOSS,BLEU。JIEBA JANOME 結果,LOSS,BLEU。
