近年來,神經機器翻譯因引入編碼器-解碼器網路而逐漸完善,但是在中文及日文的翻譯任務中,由於訓練資料的缺乏,以及和西方語言的差異性,始終無法獲得像英文與其他語言之間的高翻譯品質。本篇論文嘗試使用聲音資訊作為中日文的額外特徵,並將該特徵應用於中日文的神經機器翻譯系統當中,旨在透過以特徵工程的方式來加強翻譯品質。

本論文基於三種分詞方法,在不同的分詞下,提取出語料庫中的中文注音以及日文平假名做為聲音資訊。接著,利用文字資訊以及聲音資訊,分別訓練出帶有語義以及帶有語音的詞嵌入。我們混合兩者,產生出同時帶有語義以及語音的「合併詞嵌入」,並將其投入至兩種主流的神經機器翻譯模型,進行訓練與進一步的特徵提取。

實驗採用雙語評估替補分數 (BLEU) 對翻譯結果進行評分,結果表明,使用合併詞嵌入進行訓練與微調的模型,皆獲得比單純使用語義或語音的詞嵌入還要更高的分數;我們亦對模型的翻譯結果進行案例分析,由合併詞嵌入產生出的翻譯能夠保留正確,甚至更為精確的詞彙;也能夠保留片假名與英文的單詞,得到語義上的提升。實驗另外對合併詞嵌入和單純包含語義的詞嵌入進行四項分析,每項分析皆獲得正面回饋,顯示合併詞嵌入能夠保留,甚至超越語義詞嵌入所持有的向量涵義。

綜合翻譯分數以及詞嵌入分析,我們發現單純使用小型語料庫提取語音資訊作為特徵,便能對中日文的詞向量帶來正面影響;此外,使用合併語義及語音的詞嵌入,能夠進一步有效提升中日文神經機器翻譯系統的效能。

\begin{flushleft}
\mbox{{\bf 關鍵字}: 神經機器翻譯、詞嵌入、特徵工程、語音資訊}
\end{flushleft}