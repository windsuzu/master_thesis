近年來引入編碼器-解碼器網路而逐漸完善的神經機器翻譯,在中文及日文的翻譯任務中,由於訓練資料的缺乏、以及和西方語言的差異性,始終無法獲得像英文與其他語言之間的高翻譯品質。本篇論文嘗試使用聲音資訊作為中日文的額外特徵,並將該特徵應用於中日文的神經機器翻譯系統當中,旨在透過以特徵工程的方式來加強翻譯品質。本論文基於三個分詞方法,在不同的分詞下提取出語料庫中的中文注音以及日文平假名做為聲音資訊。接著,利用文字資訊以及聲音資訊,分別訓練出兩個單純帶有語義和語音的詞嵌入。我們合併兩者產生出同時帶有語義以及語音的合併詞嵌入,並將其投入至兩種主流的神經機器翻譯模型進行訓練與進一步的特徵提取。實驗採用雙語評估替補分數對翻譯結果進行評分,結果表明使用合併詞嵌入進行訓練與微調的模型,皆獲得比單純使用語義或語音的詞嵌入還要更高的分數;我們亦對模型的翻譯結果進行案例分析,由合併詞嵌入產生出的翻譯能夠保留正確、甚至更為精確的詞彙,也能夠保留片假名與英文的單詞,得到語義上的提升。實驗另外對合併詞嵌入和單純包含語義的詞嵌入進行四項分析,每項分析皆獲得正面回饋,顯示合併詞嵌入能夠保留乃至超越語義詞嵌入所持有的向量涵義。綜合評估翻譯分數以及詞嵌入分析,我們發現單純使用小型語料庫提取語音資訊作為特徵,便能對中日文的詞向量帶來正面影響,並且進一步使用合併語義及語音的詞嵌入能夠有效提升中日文神經機器翻譯系統的效能。

\begin{flushleft}
\mbox{{\bf 關鍵字}: 神經機器翻譯、詞嵌入、特徵工程、語音資訊}
\end{flushleft}