Neural machine translation has improved by the introduction of encoder-decoder networks in recent years. However, the translation between Chinese and Japanese has not achieved the same high quality as that between English and other languages due to the lack of training data and the differences between Eastern and Western languages. This paper attempted to use phonetic information as an additional feature in Chinese and Japanese. The aim is to enhance the translation quality by feature engineering. This paper first extracted Chinese Bopomofo and Japanese Hiragana as phonetic information from the corpus with three tokenization approaches. Second, the word embeddings with purely semantic and phonetic information were trained from the text and the phonetic information. Third, we combined both embeddings to produce a joint semantic-phonetic embedding and implemented it into two mainstream neural machine translation models for training and further extracting the feature. The results showed that the models trained and fine-tuned with joint embeddings yield higher BLEU scores than those using semantic or phonetic embeddings only. We also conducted a case study of the translation results. The translations generated with joint embeddings could produce correct and even more accurate words; and preserve the Japanese Katakana and English words, resulting in semantic improvements. Besides, four analyses conducted on joint embeddings and semantic embeddings all gave positive feedback. The feedback showed that the joint embeddings could retain and even surpass the vector meanings possessed by the semantic embeddings. Taken BLEU scores and embedding analyses together, we have found that simply using a small corpus to extract phonetic information as a feature can positively affect the Chinese and Japanese word vectors. In addition, the use of joint semantic-phonetic embeddings can effectively improve the performance of Chinese and Japanese neural machine translation systems.


\begin{flushleft}
\mbox{{\bf Keywords}: Neural Machine Translation, Word Embedding, Feature Engineering, Phoentic Information}
\end{flushleft}