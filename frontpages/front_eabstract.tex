Neural machine translation has been improved by the introduction of encoder-decoder networks in recent years. However, the translation between Chinese and Japanese has not yet achieved the same high quality as that between English and other languages due to the lack of parallel corpora and less similarity in languages. This paper attempts to use phonetic information as an additional feature for Chinese and Japanese and aims to improve translation quality through feature engineering. We first extracted Chinese bopomofo and Japanese hiragana as phonetic information from the corpus with three tokenization approaches. Second, the embedding with semantics and embedding with phonetics are generated by training separately on the text and phonetic information. Third, we combined both embeddings to create a joint semantic-phonetic embedding and implemented it into two mainstream neural machine translation models for training and further extracting the features. The results show that the models trained and fine-tuned with joint embeddings yield higher BLEU scores than those using semantic or phonetic embeddings only. We also conducted case studies on the translation results. The translations generated with joint embeddings tend to select the correct and even more precise words and retain Japanese katakana and English words, resulting in semantic improvements. Besides, four analyses conducted on joint embeddings and semantic embeddings all gave positive feedback, which showed that the joint embeddings could preserve and even outperform the vector meanings possessed by the semantic embeddings. The reveal of BLEU scores and embedding analysis demonstrates that simply using a small corpus to extract phonetic information as a feature can positively affect the Chinese and Japanese word vectors. In addition, the use of joint semantic-phonetic embeddings can effectively improve the performance of Chinese-Japanese neural machine translation systems.

\begin{flushleft}
\mbox{{\bf Keywords}: Neural Machine Translation, Word Embedding, Feature Engineering, Phonetic Information}
\end{flushleft}