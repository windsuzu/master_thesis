%% // nckuee.sty 定義 // cbj

% 產生論文封面
\nckuEEtitlepage
% 產生口試委員會簽名單
%\nckuEEoralpage
% 產生口試委員簽名單(en)
%\nckuEEenoralpage

%\newpage
%\setcounter{page}{1}
%\pagenumbering{roman}

%%%%%%%%%%%%%%%%%%%%%%%%%%%%%%%
%       封面內頁
%%%%%%%%%%%%%%%%%%%%%%%%%%%%%%%
% % unmark to add inner cover
%\newpage
%\thispagestyle{empty}
%\thispagestyle{EmptyWaterMarkPage}
%\nckuEEtitlepage


%%%%%%%%%%%%%%%%%%%%%%%%%%%%%%%
%       中文摘要
%%%%%%%%%%%%%%%%%%%%%%%%%%%%%%%

% 可以利用如下自定義的command (定義在nckuee.sty)
% ======
%\begin{zhAbstract}  %中文摘要
%中文版簡介。手動換行會自動變成下一段文字區塊。

\begin{flushleft}
\mbox{{\bf 關鍵字}: 關鍵字1、關鍵字2、關鍵字3}
\end{flushleft}
 % // 可以引入front_cabstract.tex檔案或在此編輯 // cbj
%\end{zhAbstract}

% ...等
% ======

% 在此直接定義如下
%%%%%%%%%%%%%%%%
%
\newpage
% // HongJhe 頁碼起始
\setcounter{page}{1}
\pagenumbering{roman}
% create an entry in table of contents for 中文摘要
\phantomsection % for hyperref to register this
\addcontentsline{toc}{chapter}{\nameCabstract}
% aligned to the center of the page
\begin{center}
% font size (relative to 12 pt):
% \large (14pt) < \Large (18pt) < \LARGE (20pt) < \huge (24pt)< \Huge (24 pt)
% Set the line spacing to single for the names (to compress the lines)
\renewcommand{\baselinestretch}{1}   %行距 1 倍
% it needs a font size changing command to be effective
\LARGE{\zhTitle}\\  %中文題目
\vspace{0.83cm}
% \makebox is a text box with specified width;
% option s: stretch
% use \makebox to make sure
% each text field occupies the same width
%\makebox[1.5cm][c]{\large{學生:}}
\hspace{0.5in}
\renewcommand{\thefootnote}{\fnsymbol{footnote}}
\makebox[3.5cm][l]{\large{\authorZhName\footnote[1]{}}}\footnotetext[1]{{學生}} % 學生中文姓名
%\hfill
%
%\makebox[3cm][c]{\large{指導教授:}}
\makebox[3.5cm][l]{\large{\advisorZhName\footnote[2]{}}}\footnotetext[2]{{指導教授}} \\ %指導教授中文姓名
%
\vspace{0.42cm}
%
\large{\zhUniv}\large{\zhDepartmentName}\\ %校名系所名
\vspace{0.83cm}
%\vfill
\makebox[2.7cm][c]{\large{摘要}}
\end{center}
% Resume the line spacing to the desired setting
\renewcommand{\baselinestretch}{\mybaselinestretch}   %恢復原設定
%it needs a font size changing command to be effective
% restore the font size to normal
\normalsize
%%%%%%%%%%%%%
\par  % 摘要首段空格 by SianJhe
中文版簡介。手動換行會自動變成下一段文字區塊。

\begin{flushleft}
\mbox{{\bf 關鍵字}: 關鍵字1、關鍵字2、關鍵字3}
\end{flushleft}
 % // 可以引入front_eabstract.tex檔案或在此編輯 // cbj



%%%%%%%%%%%%%%%%%%%%%%%%%%%%%%%
%       英文摘要
%%%%%%%%%%%%%%%%%%%%%%%%%%%%%%%
%
%[method 1]

% 可以利用如下自定義的command (定義在nckuee.sty)
% ======
%\begin{enAbstract}  %英文摘要
%Neural machine translation has improved by the introduction of encoder-decoder networks in recent years. However, the translation between Chinese and Japanese has not achieved the same high quality as that between English and other languages due to the lack of training data and the differences between Eastern and Western languages. This paper attempted to use phonetic information as an additional feature in Chinese and Japanese. The aim is to enhance the translation quality by feature engineering. This paper first extracted Chinese Bopomofo and Japanese Hiragana as phonetic information from the corpus with three tokenization approaches. Second, word embeddings with semantics and word embeddings with phonetics are trained based on text and phonetic information, respectively. Third, we combined both embeddings to produce a joint semantic-phonetic embedding and implemented it into two mainstream neural machine translation models for training and further extracting the feature. The results showed that the models trained and fine-tuned with joint embeddings yield higher BLEU scores than those using semantic or phonetic embeddings only. We also conducted case studies on the translation results. The translations generated with joint embeddings could produce correct and even more accurate words; and preserve the Japanese Katakana and English words, resulting in semantic improvements. Besides, four analyses conducted on joint embeddings and semantic embeddings all gave positive feedback. The feedback showed that the joint embeddings could retain and even surpass the vector meanings possessed by the semantic embeddings. Taken BLEU scores and embedding analyses together, we have found that simply using a small corpus to extract phonetic information as a feature can positively affect the Chinese and Japanese word vectors. In addition, the use of joint semantic-phonetic embeddings can effectively improve the performance of Chinese and Japanese neural machine translation systems.


\begin{flushleft}
\mbox{{\bf Keywords}: Neural Machine Translation, Word Embedding, Feature Engineering, Phoentic Information}
\end{flushleft} % // 可以引入front_eabstract.tex檔案或在此編輯 // cbj
%\end{enAbstract}

%[method 2]
\newpage
% create an entry in table of contents for 英文摘要
\phantomsection % for hyperref to register this
\addcontentsline{toc}{chapter}{\nameEabstract} % // HongJhe marked

% aligned to the center of the page
\begin{center}
% font size:
% \large (14pt) < \Large (18pt) < \LARGE (20pt) < \huge (24pt)< \Huge (24 pt)
% Set the line spacing to single for the names (to compress the lines)
\renewcommand{\baselinestretch}{1}   %行距 1 倍
%\large % it needs a font size changing command to be effective
\LARGE{\enTitle}\\  %英文題目
\vspace{0.83cm}
% \makebox is a text box with specified width;
% option s: stretch
% use \makebox to make sure
% each text field occupies the same width
%\makebox[2cm][s]{\large{Student: }}
\hspace{0.45in}
\renewcommand{\thefootnote}{\fnsymbol{footnote}}
\makebox[5cm][l]{\large{\authorEnName\footnote[1]{}}}\footnotetext[1]{{Student}} % 學生英文姓名
%\hfill
%
%\makebox[2cm][s]{\large{Advisor: }}
\makebox[5cm][l]{\large{\advisorEnName\footnote[2]{}}}\footnotetext[2]{{Advisor}} \\ %教授英文姓名
%
\vspace{0.42cm}
\large{\enDepartmentName}\\ %英文系所全名
%
\large{\enUniv}\\  %英文校名
\vspace{0.83cm}
%\vfill
%
\large{\nameEabstractc}\\
%\vspace{0.5cm}
\end{center}

% Resume the line spacing the desired setting
\renewcommand{\baselinestretch}{\mybaselinestretch}   %恢復原設定
%\large %it needs a font size changing command to be effective
% restore the font size to normal
\normalsize
%%%%%%%%%%%%%
Neural machine translation has improved by the introduction of encoder-decoder networks in recent years. However, the translation between Chinese and Japanese has not achieved the same high quality as that between English and other languages due to the lack of training data and the differences between Eastern and Western languages. This paper attempted to use phonetic information as an additional feature in Chinese and Japanese. The aim is to enhance the translation quality by feature engineering. This paper first extracted Chinese Bopomofo and Japanese Hiragana as phonetic information from the corpus with three tokenization approaches. Second, word embeddings with semantics and word embeddings with phonetics are trained based on text and phonetic information, respectively. Third, we combined both embeddings to produce a joint semantic-phonetic embedding and implemented it into two mainstream neural machine translation models for training and further extracting the feature. The results showed that the models trained and fine-tuned with joint embeddings yield higher BLEU scores than those using semantic or phonetic embeddings only. We also conducted case studies on the translation results. The translations generated with joint embeddings could produce correct and even more accurate words; and preserve the Japanese Katakana and English words, resulting in semantic improvements. Besides, four analyses conducted on joint embeddings and semantic embeddings all gave positive feedback. The feedback showed that the joint embeddings could retain and even surpass the vector meanings possessed by the semantic embeddings. Taken BLEU scores and embedding analyses together, we have found that simply using a small corpus to extract phonetic information as a feature can positively affect the Chinese and Japanese word vectors. In addition, the use of joint semantic-phonetic embeddings can effectively improve the performance of Chinese and Japanese neural machine translation systems.


\begin{flushleft}
\mbox{{\bf Keywords}: Neural Machine Translation, Word Embedding, Feature Engineering, Phoentic Information}
\end{flushleft} % // 可以引入front_eabstract.tex檔案或在此編輯 // cbj


%%%%%%%%%%%%%%%%%%%%%%%%%%%%%%%
%       誌謝
%%%%%%%%%%%%%%%%%%%%%%%%%%%%%%%
%
% Acknowledgment
\newpage
\phantomsection % for hyperref to register this
%\addcontentsline{toc}{chapter}{\nameAcknc}

\begin{zhAckn}  %誌謝
\begin{CJK}{UTF8}{ipxm}
During the months of writing the thesis, I have encountered many difficulties and obstacles, but now I feel a sense of accomplishment and warmth in my heart when I see the final draft.

First of all, I am deeply indebted to \href{https://paulhorton.gitlab.io/}{Dr. Paul Horton}. Without his guidance, support and kindness, I would never have been able to pursue the Chinese-Japanese translation system as a thesis topic. He has given a lot of foresighted but insightful advice on topic content, essay writing, etc. Every time I communicate with him, I always gain a lot. 長い間、ご指導頂きまして、本當に感謝しております。

Secondly, I would like to thank \href{https://ikmlab.csie.ncku.edu.tw/advisor.html}{Dr. Hung-Yu Kao} and \href{http://mmcv.csie.ncku.edu.tw/~wtchu/}{Dr. Wei-Ta Chu} for agreeing to be my thesis committee despite their extremely busy schedule. I am grateful to them for taking the time to read my thesis and giving it all kinds of useful advice. They are all professors who are passionate about teaching and research, and I believe their academic achievements will continue to increase.

Thirdly, I would like to thank the owners of the information, literature, and ideas cited and referenced in this thesis. Without these constructive and prospective materials, I would not have been able to accomplish my thesis. Thank you to these researchers who are willing to share their knowledge with the world.

Lastly, I would like to thank my family and all my friends. Because of your encouragement, advice, and help, I was able to complete this thesis successfully. 

\end{CJK}

\begin{flushright}
\mbox{Shih-Chieh Wang}
\end{flushright} % // 可以引入front_ackn.tex檔案或在此編輯 // cbj
\end{zhAckn}

%\chapter*{\nameAckn} %\makebox{} is fragile; need protect
%\begin{CJK}{UTF8}{ipxm}
During the months of writing the thesis, I have encountered many difficulties and obstacles, but now I feel a sense of accomplishment and warmth in my heart when I see the final draft.

First of all, I am deeply indebted to \href{https://paulhorton.gitlab.io/}{Dr. Paul Horton}. Without his guidance, support and kindness, I would never have been able to pursue the Chinese-Japanese translation system as a thesis topic. He has given a lot of foresighted but insightful advice on topic content, essay writing, etc. Every time I communicate with him, I always gain a lot. 長い間、ご指導頂きまして、本當に感謝しております。

Secondly, I would like to thank \href{https://ikmlab.csie.ncku.edu.tw/advisor.html}{Dr. Hung-Yu Kao} and \href{http://mmcv.csie.ncku.edu.tw/~wtchu/}{Dr. Wei-Ta Chu} for agreeing to be my thesis committee despite their extremely busy schedule. I am grateful to them for taking the time to read my thesis and giving it all kinds of useful advice. They are all professors who are passionate about teaching and research, and I believe their academic achievements will continue to increase.

Thirdly, I would like to thank the owners of the information, literature, and ideas cited and referenced in this thesis. Without these constructive and prospective materials, I would not have been able to accomplish my thesis. Thank you to these researchers who are willing to share their knowledge with the world.

Lastly, I would like to thank my family and all my friends. Because of your encouragement, advice, and help, I was able to complete this thesis successfully. 

\end{CJK}

\begin{flushright}
\mbox{Shih-Chieh Wang}
\end{flushright} % // 可以引入my_ackn.tex檔案或在此編輯 // cbj
%%testjsjtoejiojsoijtoijos

%%%%%%%%%%%%%%%%%%%%%%%%%%%%%%%
%       目錄
%%%%%%%%%%%%%%%%%%%%%%%%%%%%%%%
%
% Table of contents
\newpage
\renewcommand{\contentsname}{\nameToc}
%\makebox{} is fragile; need protect
\phantomsection % for hyperref to register this
\addcontentsline{toc}{chapter}{\nameTocc}
\begin{spacing}{1.5}
\tableofcontents
\end{spacing}

%%%%%%%%%%%%%%%%%%%%%%%%%%%%%%%
%       表目錄
%%%%%%%%%%%%%%%%%%%%%%%%%%%%%%%
%
% List of Tables
\newpage
\renewcommand{\listtablename}{\nameLot}
%\makebox{} is fragile; need protect
\phantomsection % for hyperref to register this
\addcontentsline{toc}{chapter}{\nameLotc}
\listoftables

%%%%%%%%%%%%%%%%%%%%%%%%%%%%%%%
%       圖目錄
%%%%%%%%%%%%%%%%%%%%%%%%%%%%%%%
%
% List of Figures
\newpage
\renewcommand{\listfigurename}{\nameTof}
%\makebox{} is fragile; need protect
\phantomsection % for hyperref to register this
\addcontentsline{toc}{chapter}{\nameTofc}
\listoffigures
%%%%%%%%%%%%%%%%%%%%%%%%%%%%%%%
%       符號說明
%%%%%%%%%%%%%%%%%%%%%%%%%%%%%%%
%
% Symbol list
% define new environment, based on standard description environment
% adapted from p.60~64, <<The LaTeX Companion>>, 1994, ISBN 0-201-54199-8

%\newcommand{\SymEntryLabel}[1]%
%  {\makebox[3cm][l]{#1}}
%%
%\newenvironment{SymEntry}
%   {\begin{list}{}%
%       {\renewcommand{\makelabel}{\SymEntryLabel}%
%        \setlength{\labelwidth}{3cm}%
%        \setlength{\leftmargin}{\labelwidth}%
%        }%
%   }%
%   {\end{list}}
%%%
%\newpage
%\chapter*{\nameSlist} %\makebox{} is fragile; need protect
%\phantomsection % for hyperref to register this
%\addcontentsline{toc}{chapter}{\nameSlistc}
%%
% this file is encoded in utf-8
% v2.0 (Apr. 5, 2009)
%  各符號以 \item[] 包住,然後接著寫說明
% 如果符號是數學符號,應以數學模式$$表示,以取得正確的字體
% 如果符號本身帶有方括號,則此符號可以用大括號 {} 包住保護
\begin{SymEntry}

\item[OLED]
Organic Light Emitting Diode

\item[$E$]
energy

\item[$e$]
the absolute value of the electron charge, $1.60\times10^{-19}\,\text{C}$
 
\item[$\mathscr{E}$]
electric field strength (V/cm)

\item[{$A[i,j]$}]
the  element of the matrix $A$ at $i$-th row, $j$-th column\\
矩陣 $A$ 的第 $i$ 列,第 $j$ 行的元素

\end{SymEntry}

\newpage
\setcounter{page}{1}
\pagenumbering{arabic}
