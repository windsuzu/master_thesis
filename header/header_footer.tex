%
% this file is encoded in utf-8
% v2.0 (Apr. 5, 2009)

%%%%%%% 其他的 header (left, right) 定義
% 底下定義了一些常見的 header 型式
% 預設情況是關掉的
% 使用者可以視需要將之打開
% 也就是把下列介於 >>> 與 <<< 之間
% 的文字行打開 (行首去掉百分號)

%% header >>>
%\renewcommand{\chaptermark}[1]{%
%\markboth{\prechaptername\ \thechapter\ \postchaptername%
%\ #1}{}%
%}  %定義 header 使用的「章」層級的戳記
%\fancyhead[L]{} % 左 header 為空
%\fancyhead[R]{\leftmark}  % 右 header 擺放「章」層級的戳記 (以 \leftmark 叫出)
%\renewcommand{\headrulewidth}{0.4pt}  % header 的直線 0.4pt; 0pt 則無線
%% <<< header

%%%%%%% 其他的 footer (left, right) 定義
% 底下定義了一些常見的 footer 型式
% 預設情況是關掉的
% 使用者可以視需要將之打開
% 也就是把下列介於 >>> 與 <<< 之間
% 的文字行打開 (行首去掉百分號)

%% footer >>>
% \fancyfoot[L]{} % 左 footer 為空
% \fancyfoot[R]{\small{YZU \LaTeX\ v2.0}} % 右 footer 擺放論文格式版本
% \renewcommand{\footrulewidth}{0.4 pt} % footer 的直線 0.4pt; 0pt 則無線
%% <<< footer
