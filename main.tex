\documentclass[12pt,p a4paper]{class/ncku_class}

\usepackage{times}
\usepackage{CJKutf8}  %%% ZZZ %%% macro for Chinese/Japanese/Korean processing
\usepackage{CJKnumb} %%% ZZZ %%% for Chinese numbering capability

\usepackage[nospace]{cite}  % for smart citation
\usepackage{geometry}  % for easy margin settings
\usepackage{class/ncku_style} % 自定義nckuee.sty
\usepackage{booktabs}
\usepackage[pdftex]{graphicx}
\newcommand\myworkflow{pdftex}  % set the flag
\usepackage{fancyhdr}  % 借用此套件來擺放浮水印

% 啟動 fancy header/footer 套件
\pagestyle{fancy}
\fancyhead{}  % reset left, central, right header to empty
\fancyfoot[C]{\thepage} %中間 footer 擺放頁碼
\renewcommand{\headrulewidth}{0pt} % header 的直線; 0pt 則無線

% 如果不需要任何浮水印,則請把下列介於 >>> 與 <<< 之間
% 的文字行關掉 (行首加上百分號)
%% 浮水印 >>>
\input{header/ncku_watermark.tex}
%% <<< 浮水印
% 如需額外的頁楣 (header) 或 footer,請在 header/header_footer.tex 裡依例修改
% 它的預設內容是都關掉,可依需要打開
\input{header/header_footer.tex}

\usepackage{amsmath} % 各式 AMS 數學功能
\usepackage{amssymb} % 各式 AMS 數學符號
\usepackage{mathrsfs} %草寫體數學符號,在數學模式裡用 \mathscr{E} 得草寫 E
\usepackage{listings} % 程式列表套件
\usepackage{subfigure}
\usepackage{graphicx} \graphicspath{{images/}}
\usepackage{psfrag} % text replacement in figure
\usepackage{bm}
\usepackage{mathtools}
\usepackage{hyperref}
\hypersetup{
colorlinks,%
citecolor=blue,%
filecolor=blue,%
linkcolor=blue,%
urlcolor=blue%
}

\usepackage{setspace}
\usepackage{array}
\usepackage{threeparttable} % table note
\usepackage{multirow} % newline on the tabular of a table.
\usepackage{tabularx}
\usepackage{url}
\usepackage[usenames,dvipsnames]{xcolor}
\usepackage{pgf}
\usepackage{tikz}
% \usetikzlibrary{arrows,automata,positioning}
\usetikzlibrary{arrows,shapes}

% listing setting
\lstset{
breaklines=true,% 過長的程式行可斷行
extendedchars=false,% 中文處理不需要 extendedchars
texcl=true,% 中文註解需要有 TeX 處理過的 comment line, 所以設成 true
comment=[l]\%\%,% 以雙「百分號」做為程式中文註解的起頭標記,配合 MATLAB
basicstyle=\small,% 小號字體, 約 10 pt 大小
commentstyle=\upshape,% 預設是斜體字,會影響註解裏的英文,改用正體
%language=Octave % 會將一些 octave 指令以粗體顯示
}

\usepackage{url} % 在文稿中引用網址,可以用 \url{http://www.yzu.edu.tw} 方式

% hyperref 會擾亂 cite.sty 對文獻號碼縮編的排版,所以依據
% http://www.ctan.org/tex-archive/macros/latex/contrib/hyperref/
% 作如下的更動,使得 hyperref 不做文獻號碼的超連結。
\makeatletter
\def\NAT@parse{\typeout{This is a fake Natbib command to fool Hyperref.}}
\makeatother

% hyperlinkable table of contents 章節目錄、圖表超連結
\usepackage{hyperref}
\hypersetup{colorlinks, citecolor=blue,filecolor=blue,linkcolor=blue,urlcolor=blue}

% hyperref跟algorithm衝突,hyperref必須放在algorithm前面
\usepackage{algorithm}
\usepackage{algpseudocode}
\usepackage{enumerate}

\begin{document}
\begin{CJK}{UTF8}{bkai}
\ifx\myworkflow\mydvipdfmxflow
	\DeclareGraphicsExtensions{.pdf,.png,.jpg,.eps}
	\DeclareGraphicsRule{.pdf}{eps}{.bb}{}
	\DeclareGraphicsRule{.png}{eps}{.bb}{}
	\DeclareGraphicsRule{.jpg}{eps}{.bb}{}
\fi

% global CJK setting
\CJKindent  %%% ZZZ %%%  段首內縮兩格

% 載入中文名詞的定義:例如,Figure -->「圖」, Chapter -->「第 x 章」
\input{header/ncku_definitions.tex} % 主標題名稱定義 // 摘要 (Abstract), ... 等

\input{frontpages/ncku_frontpages.tex}
%\usepackage{geometry}  % for easy margin settings
%% margins setting // NCKU內頁設定 // cbj
%\geometry{verbose,a4paper,tmargin=2.3cm,bmargin=3.5cm,lmargin=2.5cm,rmargin=3cm}

\chapter{Introduction} \label{ch:introduction}
	\hspace{24pt}

Over the past few years, the field of neural machine translation (NMT) between Chinese and Japanese is still an unresolved problem. Recent studies in Chinese-Japanese NMT have used specific methods such as sub-character level features to improve the translation quality. This is due to the lack of parallel corpus and the difference between logogram (a character or symbol that represents a word) and alphabet (a set of letters used when writing in a language) writing systems. This research explores phonetic information as an additional feature for improving the quality of Chinese-Japanese NMT systems.

\section{Background} \label{sec:background}

NMT is a popular area of natural language processing (NLP), has been proposed by using an end-to-end model which transforms a source sentence into a latent space and decodes it directly into a target sentence \cite{sutskever2014sequence, cho2014learning}. The model is called the encoder-decoder model or sequence-to-sequence model, and they are widely used by large technology companies such as Google, Facebook, Microsoft, and DeepL.

\subsection{Progress of Neural Machine Translation} \label{sec:nmt}

The progress of NMT and NLP are inseparable. The development of models, tokenization methods, embeddings, and the solutions to less or no parallel data, all involved in the progress of NMT.

In the development of models, recurrent neural network (RNN) was first applied in NMT research \cite{sutskever2014sequence, cho2014learning}. After that, \cite{bahdanau2014neural} designed a mechanism called attention and addressed the problem of insufficient information in the latent space between encoder and decoder based on RNN. The structure of the Transformer was later proposed by \cite{NIPS2017_3f5ee243}, which replace the RNN structure with a full attention mechanism (i.e., self-attention) to achieve better results and was widely used in NMT tasks. This paper will use the attention-based RNN model and Transformer \cite{bahdanau2014neural,NIPS2017_3f5ee243} as the baseline system for the experiment.

Tokenization is one of the most important parts of any NLP task. It determines how a sentence will be tokenized, and it will generate different meanings to a sentence with different algorithms. Besides word-level and character-level tokenization, several subword-level tokenization algorithms had become the mainstream. For example: Byte-Pair Encoding (BPE) \cite{sennrich_neural_2016}, Unigram Language Model \cite{kudo-2018-subword}, WordPiece \cite{6289079}, and SentencePiece \cite{kudo-richardson-2018-sentencepiece}. This paper will utilize BPE, SentencePiece \cite{sennrich_neural_2016, kudo-richardson-2018-sentencepiece} and two word-level tokenizer (\textit{Jieba}\footnote{https://github.com/fxsjy/jieba} and \textit{Janome}\footnote{https://mocobeta.github.io/janome}) as tokenization methods.

The concept of embeddings, also known as distributed representations, was first proposed by \cite{hinton1986learning, bengio2003neural}, but was difficult to implement due to hardware limitations. With the development of parallel computing and GPU, many embedding implementations have been proposed, such as Word2Vec \cite{mikolov2013distributed}, GloVe \cite{pennington2014glove}, and fastText \cite{bojanowski2017enriching}. The contextualized word embedding is another concept that obtains context-dependent word embedding from the whole sentence, meaning that the same word with different position can obtain different embedding through the model. The representative ones are ELMo \cite{peters-etal-2018-deep} and BERT \cite{devlin-etal-2019-bert}. This paper will select Word2Vec \cite{mikolov2013distributed} as the tool for creating word embeddings because of its simplicity, rapidity, and convenience of analysis.

Several fields have been studied to solve the problems like low-resources and noisy parallel data in NMT tasks. Back-translation \cite{sennrich-etal-2016-improving} is a data augmentation method that uses monolingual data of the target language to generate source data and offset the imbalance between encoder and decoder. Parallel corpus filtering was examined for a large number of NMT tasks \cite{koehn2018findings}, using pre-filtering rules and scoring functions to retain good sentence pairs can effectively reduce the corpus size and obtained better translation results. This paper will practice corpus filtering to retain quality training data and reduce corpus size to increase experimental efficiency.

\subsection{Chinese-Japanese Neural Machine Translation}

NMT system has gained a lot of improvement in translating between English and other languages by utilizing the techniques described in section \ref{sec:nmt}. However, the improvement in translating between Chinese and Japanese is limited. The main reasons are the inadequacy of the corpus and the differences in the writing systems of Chinese, Japanese, and Western languages.

Many studies have focused on improving the Chinese-Japanese (zh-ja) NMT system. In addition to using the methods \cite{imamura2018enhancement, chu2017empirical, zhang2020parallel} described in section \ref{sec:nmt}, many feature engineering techniques have been proposed to utilize the features in Chinese Characters (\textit{Hanzi}) and Japanese \textit{Kanji}. For example, a character-level zh-ja NMT system had been improved by using radicals as character feature information \cite{8300572}. Furthermore, the use of decomposed sub-character level information such as ideographs and strokes of Chinese characters, also improved the results \cite{zhang-komachi-2018-neural}.

\subsection{Phonetic Information} \label{sec:phonetic}

Phonetic information is another feature that had been applied to NMT systems. \cite{khan2019diversity} had suggested that a phonetic representation usually corresponds to semantically distinct characters or words. \cite{liu-etal-2019-robust} had pointed out that phonetic information can effectively resist the homophone noises generated by typographical mistakes in Chinese sentences. Both papers had improved the performance of the NMT system between Chinese and other Western languages.

This paper attempts to use \textit{Bopomofo} and \textit{Hiragana} as Chinese and Japanese phonetic information to improve the performance of the zh-ja NMT system. Bopomofo also named \textit{Zhuyin} (注音), is located in the Unicode block in the range U+3100–U+312F. It consists of 37 characters and 4 tone marks to transcribe all possible Chinese characters. Although it is the main component of Mandarin Chinese, it usually does not appear in Chinese sentences. That is, the machine loses some of the phonetic information when reading Chinese sentences. \begin{CJK}{UTF8}{song}
Hiragana (平仮名, ひらがな) is a component of Japanese, along with \textit{Katakana} and Kanji. It consists of 46 base characters and is located in the Unicode block in the range U+3040–U+309F. Compared to Bopomofo, Hiragana is often found in Japanese sentences with Katakana and Kanji, forming mixed writing of Kanji and Kana (仮名交じり文). However, Hiragana disappears after forming Kanji, just like Bopomofo forms Hanzi. Therefore, the machine cannot obtain the phonetic information directly from Japanese sentences.
\end{CJK}


\section{Objective} \label{sec:objective}

This paper aims to determine whether the use of phonetic information can help improve the performance of the zh-ja NMT system. We will use embedding, which is commonly used to represent semantics, to represent the features of phonetic information. The \textit{gensim} library \footnote{https://radimrehurek.com/gensim/index.html} will be utilized to implement Word2Vec \cite{mikolov2013distributed} to extract both semantic and phonetic embedding. The embeddings will be trained on a small corpus (less than 1 million lines of sentences) to see if they are useful for the subsequent NMT task.

Combining the findings from other studies \cite{liu-etal-2019-robust, khan2019diversity} described in section \ref{sec:phonetic}, we hypothesize that embeddings with the combination of semantics and phonetics (joint embedding) can improve the performance of the zh-ja NMT system more effectively than embeddings with only semantics or phonetics. 

We will perform a series of experiments to test the hypothesis. First, we examine whether the joint embedding can improve the results of the zh-ja NMT system with different tokenization methods. Second, we conduct NMT tasks under four conditions: without any pre-trained embedding, with pre-trained semantic embedding, with pre-trained phonetic embedding, and with joint semantic-phonetic embedding. Third, we analyze the changes that occur when phonetic information is added to the general semantic embedding. The analyses include analogy reasoning, outlier detection, word similarity, and the influence on both homonyms and heteronyms.

\section{Related Work} \label{sec:related_work}

The core technique in this paper is to enhance the ability of word embeddings by utilizing feature engineering on phonetic information. Therefore, we are going to review some studies that use additional features to improve word embeddings. The review will be carried out from two perspectives, one is to improve embedding with the features of Chinese characters such as radicals and strokes, and the other is to improve embedding with the features of phonetics. The concept of Word2Vec \cite{mikolov2013distributed} is commonly used in reviews, and we will mention it in section \ref{sec:word2vec} of Chapter \ref{ch:method}.

\subsection{Chinese Word Embedding} \label{sec:rw_cwe}

We will review some studies which suggested that the rich, decomposed information of Chinese characters can be exploited to train the word embeddings with words or characters jointly. Some of the most popular studies in this field are: \textit{CWE} \cite{chen2015joint}, \textit{MGE} \cite{yin2016multi}, \textit{JWE} \cite{yu2017joint}, and \textit{cw2vec} \cite{cao2018cw2vec}. The following sections review the approaches from JWE and cw2vec because their concepts are relatively new and their performance is better.

\subsubsection{Joint Learning Word Embedding Model (JWE)}

The authors of JWE \cite{yu2017joint} proposed a model that combines word, character, and sub-character components to learn embeddings. They implemented the model (Figure \ref{fig:jwe}) based on the Continuous Bag of Words (CBOW) proposed in Word2Vec \cite{mikolov2013distributed}, and changed the default input words in CBOW by further adding the characters and sub-character components of input words.

\begin{figure}[h]
	\centering
	\includegraphics[scale=0.7]{../images/jwe_illustration.png}
	\caption{An illustration of JWE model proposed in \cite{yu2017joint}}
	\label{fig:jwe}
\end{figure}

As shown in the Figure \ref{fig:jwe} above, The JWE model is trained to predicts the present word from the context words in the same way as CBOW. The symbol $w$ represents words, $w_i$ is the current word, $w_{i-1}$ and $w_{i+1}$ are the context words. The symbol $c$ represents the characters of context words, $c_{i-1}$ is the characters of $w{i-1}$, and $c_{i+1}$ is the characters of $c_{i+1}$. The symbol $s$ represents the sub-characters of context words. $s_i$ is the sub-character components of $w_i$, and $s_{i-1}, s_{i+1}$ are the components of $w_{i-1}$ and $w_{i+1}$. For example, a word 智能 (intelligence) has two characters 智 (wisdom) and 能 (able), and each has sub-character components: 知, 日 and ㄙ, 月, 匕, 匕. The JWE model aims to maximize the sum of log-likelihoods of 3 conditional probabilities that come from the context words ($h_{i1}$), characters ($h_{i2}$), and sub-characters ($h_{i3}$).

\begin{equation}
L(w_i) = \sum_{k=1}^3\log P(w_i\mid h_{ik})    
\end{equation}

\subsubsection{cw2vec Model}

Inspired by fastText \cite{bojanowski2017enriching}, The authors of cw2vec \cite{cao2018cw2vec} proposed an n-gram feature based on the strokes of Chinese characters. They first split and transformed the text into stroke information, and mapped the strokes into 5 corresponding stroke ids, then merge the ids and generate n-gram features from these stroke ids. The diagram in the paper (Figure \ref{fig:cw2vec1}) shows the complete process.

\begin{figure}[h]
	\centering
	\includegraphics[scale=0.6]{../images/cw2vec_illustration1.png}
	\caption{The Process of generating stroke n-grams shown in \cite{cao2018cw2vec}}
	\label{fig:cw2vec1}
\end{figure}

The authors used Skip-Gram, a second method other than CBOW proposed by Word2Vec \cite{mikolov2013distributed}, as the base model, and replaced word inputs with stroke n-grams (Figure \ref{fig:cw2vec2}). It is mentioned in the paper that the final embeddings come from the contextual word vectors.

\begin{figure}[h]
	\centering
	\includegraphics[scale=0.6]{../images/cw2vec_illustration2.png}
	\caption{An illustration of cw2vec model proposed in \cite{cao2018cw2vec}}
	\label{fig:cw2vec2}
\end{figure}

These word embeddings designed for Chinese all utilized the radicals or strokes in Chinese characters. No research has so far used phonetic information extracted from Chinese or Japanese as features.

\subsection{Phonetic Word Embedding} \label{sec:rw_pwe}

We will review studies that use phonetic information to construct word embeddings. These studies are closely related to our research, and we will learn from their practical differences and use them as the cornerstone for our experiments.

\subsubsection{Phonetic Encoding}

The study \cite{khan2019diversity} extracted sentences from Chinese or Western languages into phonetic encodings \footnote{The logogram encoding is also used as a feature but is not explained here. Random clustering is used for comparison purposes.}, which used \textit{Soundex}, \textit{NYSIIS}, \textit{Metaphone}, and \textit{Hanyu Pinyin} as the extraction algorithm. After that, they applied BPE \cite{sennrich_neural_2016} to tokenize the sentences and encodings, and trained seperate embeddings from the sentences and each encodings (empty boxes in Figure \ref{fig:phonetic1}). Lastly, they concatenated the embeddings and fed into NMT system. 

\begin{figure}[h]
	\centering
	\includegraphics[scale=0.9]{../images/phonetic_encoding.png}
	\caption{Implementation of phonetic encoding proposed in \cite{khan2019diversity}}
	\label{fig:phonetic1}
\end{figure}

This paper did not explain in detail how to implement phonetic information in embedding, but they had presented a hypothesis and verified it. That is, phonetics is a function that groups semantically distinct words. The words with the same pronunciation and spelling are usually distinguished by the different contexts.

\newpage

\subsubsection{Joint Textual and Phonetic Embedding}

This paper \cite{liu-etal-2019-robust} focused on using phonetic embedding to adjust the homophone noise problem, which is a frequent problem in a parallel corpus. For instance, when a single word in the source sentence is misplanted as a corresponding homophone (e.g., 有 (yǒu, to have) is wrongly replaced with 友 (yǒu, friend)), the model will read the wrong embedding and training in a wrong direction. The phonetic embedding can be seen as a feature to offset the errors that occur in semantic embeddings with homophone noise.

The paper showed that they trained the text (denoted by $a$) as semantic embedding (denoted by $\pi(a)$) and phonetic embedding (denoted by $\psi(a)$) respectively, and then combined two embeddings with a configurable parameter (denoted by $\beta$) as follows:

\begin{equation}
	\pi([a, \psi(a)]) = (1-\beta) \times \pi(a) + \beta \times \pi(\psi(a)) 
\end{equation}

This study did not give the details of constructing embeddings, but they had mentioned that the best result occurs when the $\beta$ is $0.95$. That is, when they used 5\% of semantic embedding and 95\% of phonetic embedding, they obtained the best performance in their NMT system.

To summarize the findings in these related works. The phonetic encodings can emphasize the difference between semantically diverse sentences. The joint semantic-phonetic embedding also shows the robustness to noise in parallel corpora. All of these features can help improve the performance of the NMT system. However, the methods of using phonetic information as embeddings had been explored mainly in Chinese and Western languages. They also used \textit{Pinyin} instead of Bopomofo as the component to decompose Chinese characters. According to our study, no research has proposed to apply and analyze phonetic information as an additional feature in Chinese and Japanese NMT systems.


\chapter{Method} \label{ch:method}
	\hspace{24pt}

method.

\chapter{Experiment and Result} \label{ch:experiment_and_result}
	\hspace{24pt}

experiment and result.

\section{Environment} \label{sec:environment}
\subsection{PyTorch Lightning} \label{sec:lightning}
\subsection{wandb} \label{sec:wandb}

\section{Dataset} \label{sec:dataset}

\section{Parameter} \label{sec:parameter}

\section{Metric} \label{sec:metric}

\section{Result} \label{sec:result}

\chapter{Discussion} \label{ch:discussion}
	\hspace{24pt}

discussion.

\section{Case Study} \label{sec:case_study}


\section{Embedding Analysis} \label{sec:analysis}


\chapter{Conclusion} \label{ch:conclusion}
	\hspace{24pt}

% 我們將總結整個論文所做的假設、方法、貢獻。講出本論文能進步、加強的地方。

% 這個章節首先將在a總結論文所提出的假設,方法,及貢獻。接著在b提出一系列可以深入研究或改進的地方,作為我們的未來規劃。

\section{Conclusion} \label{sec:conclusion}

% 假設、方法、貢獻

% 本篇論文嘗試使用聲音資訊作為文字的額外特徵,並將該特徵應用於中文與日文之間的神經機器翻譯系統當中,旨在透過常規以外的方法如 back translation 或 model ensembling ,以 feature engineering 的方式來加強翻譯系統。 我們提出使用三種不同的分詞方法,以 dragonmapper 及 pykakasi 分別產生中文及日文的聲音資訊,並搭配 word2vec 訓練出兩個單純帶有語意和音意的詞嵌入。我們合併兩者產生帶有語意及音意的合併詞嵌入,並將其投入至兩種知名的 NMT 模型進行訓練與fine-tune。結果表明所有使用了合併詞嵌入進行訓練與fine-tune的模型,其獲得的 BLEU 分數都比其他方法還要更高。我們透過對最好模型的結果進行 case study 發現使用合併詞嵌入翻譯的結果保留了四種property。我們透過對合併詞嵌入與單純包含詞義的嵌入進行分析,結果也表明在四種任務下皆保留原有效果,甚至能超越詞義嵌入,獲得非常正面的反饋。總結,該論文提出的方法能有效提升nmt效能,而且僅使用小型語料庫。對於翻譯結果與詞嵌入我們也展示一系列的分析,得知該詞嵌入確實對於中日文翻譯有明顯的幫助。

\section{Future Work} \label{sec:future_work}

% 本論文有許多能夠深入研究的方向,我們在此列出幾個方法,這些方法有的或許能進一步分析現有表現,有的或許有機會能進一步提高表現。

% 改善產生聲音資訊的方法,例如使用其他的插件來產生並比較優劣,舉例

% 採用在 related work 中提到訓練 embedding 的模式,將聲音資訊與中文字、及各種 subcharacter feature 例如筆畫、部首來一起訓練一個 embedding。

% 採用 ELMO、BERT 當作基礎的 embedding 架構。由於他們兩者都是屬於需要綁定模型的 embedding,所以可以考慮訓練一個 semantic model 和一個 phonetic model 並將兩者結合起來。

% 最新研究展示 cnn 用於 NLP 領域強於 transformer。所以我們也可以試用 CONVS2S 當作模型架構,。
\newpage
\phantomsection % for hyperref to register this
\addcontentsline{toc}{chapter}{References}
\renewcommand{\bibname}{\protect\makebox[5cm][s]{REFERENCES}}
%  \makebox{} is fragile; need protect
\bibliographystyle{apalike}
% \bibliographystyle{IEEEtranS}  % 使用 IEEE Trans 期刊格式

%% References with bibTeX database:
\bibliography{back/ncku_thesis_bib}
\baselineskip=18pt



\clearpage % to make sure all CJK characters are processed
\end{CJK}
\end{document}